%%==============================================================
%% Autor: Matheus dos Reis de Jesus
%% Última versão Maio/2021
%% Arquivo em formato UTF-8
%% Compilar com pdftex
%% Precisa do arquivo abntex2-UFV.sty e abntex2cite.tex
%%
%% Modelo: Rodrigo Smarzaro (smarzaro@ufv.br)
%%==============================================================

\documentclass[
	% -- opções da classe memoir --
	12pt,				    % tamanho da fonte
	openright,			    % capítulos começam em pág ímpar (insere página vazia caso preciso)
	oneside,			    % para impressão só no anverso. Oposto a twoside
	a4paper,			    % tamanho do papel.
    % -- opções do pacote abntex2 --
    % chapter=TITLE,         % Títulos em maiúsculas
    sumario=tradicional,    % Sumário padrão memoir (mais bonito "imo")
    % -- opções do pacote babel --
	english,			    % idioma adicional para hifenização
	brazil,				    % o último idioma é o principal do documento
	]{abntex2}              % Personaliza a capa. Precisa do arquivo ufv.cls para funcionar.


% Pacotes fundamentais
\usepackage{abntex2-UFV}        % Personalização para a Universidade Federal de Viçosa
\usepackage{lmodern}			% Usa a fonte Latin Modern			
\usepackage[T1]{fontenc}		% Selecao de codigos de fonte de saída
\usepackage[utf8]{inputenc}		% Codificacao do documento (conversão automática dos acentos)
\usepackage{indentfirst}		% Indenta o primeiro parágrafo de cada seção.
\usepackage{graphicx}			% Inclusão de gráficos
\usepackage{float}				% Posicionamento específico de figuras
\usepackage{booktabs}           % \toprule, \midrule e \bottomrule para tabelas
\usepackage{hyperref}
% Sistema autor-data com títulos nas referências em negrito
% \usepackage[abbrv,abnt-emphasize=bf]{abntex2cite}
\usepackage[backend=biber, style=numeric]{biblatex}
\addbibresource{referencias.bib}

% ---
% CONFIGURAÇÕES DE PACOTES
% ---

% Informações de dados para CAPA e FOLHA DE ROSTO
\titulo{\emph{Desenvolvimento de um UI Framework em React Native com componentes focados em acessibilidade para idosos em dispositivos móveis}}
\autor{Matheus dos Reis de Jesus}
\local{Viçosa}
\data{2021}
\orientador{Lucas Francisco da Matta Vegi}    % redefinido no abntex2-UFV para aceitar Instituição (default = UFV-CRP)
%\coorientador{Nome do Coorientador}
\instituicao{Universidade Federal de Viçosa}

\campus{\emph{Campus} de Viçosa}      % pacote abntex2-UFV
\curso{Ciência da Computação}               % pacote abntex2-UFV
%\membrobancaA{Membro da Banca A}             % pacote abntex2-UFV default = UFV-CRP
%\membrobancaB[UFMG]{Membro da Banca B}       % pacote abntex2-UFV default = UFV-CRP
%\databanca{\today}                           % pacote abntex2-UFV

% O preambulo deve conter o tipo do trabalho, o objetivo,
% o nome da instituição e a área de concentração
\preambulo{Projeto apresentado à Universidade Federal de Viçosa como parte das exigências para a aprovação na disciplina Seminários II}
% ---

% ---
% Configurações de aparência do PDF final

% informações para o arquivo pdf de saída
% Interessante alterar a cor dos links para preto(black)
% para imprimir
\makeatletter
\hypersetup{
        % metadados
		pdftitle={\@title},
		pdfauthor={\@author},
    	pdfsubject={\imprimirpreambulo},
	    pdfcreator={LaTeX with abnTeX2},
		colorlinks=true,   % false: links em frame; true: links coloridos
    	linkcolor=black,    % cor dos links no documento
    	citecolor=blue,    % cor dos links para a bibliografia
    	filecolor=magenta, % cor dos links para arquivos
		urlcolor=blue,     % cor dos links para sites
		bookmarksdepth=4   % profundidade do sumário do PDF
}
\makeatother
% ---

\begin{document}
% Retira espaço extra obsoleto entre as frases.
\frenchspacing

% ----------------------------------------------------------
% ELEMENTOS PRÉ-TEXTUAIS
% ----------------------------------------------------------
\pretextual

% Capa
\imprimircapa

% Folha de rosto
\imprimirfolhaderosto
% ---

% Inserir folha de aprovação
%\imprimirfolhadeaprovacao

% Dedicatória
%\begin{dedicatoria}
%   \vspace*{\fill}
%   \centering
%   \noindent
%   \textit{Texto qualquer da dedicatória}
%   \vspace*{\fill}
%\end{dedicatoria}
% ---

% Agradecimentos
%\begin{agradecimentos}

%\end{agradecimentos}
% ---

% Epígrafe
%\begin{epigrafe}
%    \vspace*{\fill}
%	\begin{flushright}
%		\textit{``Word? nunca mais.''\\
%		(Qualquer usuário de \LaTeX)}
%	\end{flushright}
%\end{epigrafe}
% ---

% RESUMOS

% resumo em português
\begin{resumo}
	\noindent

	\MakeUppercase{\textbf{Desenvolvimento de um UI Framework em React Native com componentes focados em acessibilidade para idosos em dispositivos móveis}}

	\vspace{\onelineskip}
	Lucas Francisco da Matta Vegi (Orientador) \newline
	Matheus dos Reis de Jesus (Estudante)
	\vspace{\onelineskip}

	\MakeUppercase{\textbf{Resumo}}

	A constante evolução da tecnologia tem trazido cada vez mais recursos inovadores que têm transformado a maneira como interagimos com o mundo. De um computador de mesa à um dispositivo móvel, a quantidade de recursos que temos à nossa disposição é quase imensurável. Porém, esses recursos têm suas limitações, principalmente por alcançarem públicos diversos e não necessariamente estarem preparados para lidar com as necessidades que estes possuem. Especialmente em dispositivos móveis, grande parte das aplicações não possuem recursos que possibilitem o uso por pessoas com necessidades especiais, como por exemplo pessoas idosas. Tendo isso em mente, o presente projeto foi idealizado para fornecer recursos que possam sanar as principais dificuldades enfrentadas por idosos ao tentar utilizar um dispositivo móvel.

	\vspace{\onelineskip}

	\noindent
	\MakeUppercase{\textbf{Palavras-chaves}} \newline
	Acessibilidade, React Native, Framework, UI
	\vspace{\onelineskip}

	\noindent
	\MakeUppercase{\textbf{Área de Conhecimento}} \newline
	1.03.03.04-9 Sistemas de Informação
	\vspace{\onelineskip}

	\noindent
	\MakeUppercase{\textbf{Linha de Pesquisa}} \newline
	DPI-016 – Sistemas de Informação
\end{resumo}

% resumo em inglês
%\begin{resumo}[Abstract]
% \begin{otherlanguage*}{english}
%   \noindent
%   % Insira o abstract aqui
%
%   \vspace{\onelineskip}
%
%   \noindent
%   \textbf{Key-words}: TCC, latex, abntex, UFV..
% \end{otherlanguage*}
%\end{resumo}

% inserir lista de ilustrações
\pdfbookmark[0]{\listfigurename}{lof}
\listoffigures*
\cleardoublepage
% ---

% inserir lista de tabelas
\pdfbookmark[0]{\listtablename}{lot}
\listoftables*
\cleardoublepage
% ---


% inserir o sumario
\pdfbookmark[0]{\contentsname}{toc}
\tableofcontents*
\cleardoublepage
% ---

% ----------------------------------------------------------
% ELEMENTOS TEXTUAIS
% ----------------------------------------------------------
\textual

\chapter{Introdução}\label{sec:introducao}

Segundo a \emph{30ª Pesquisa Anual de Uso de TI},
realizada pela \emph{Fundação Getúlio Vargas}\cite{pesquisati30}, em abril de 2019 o Brasil já contava com aproximadamente 230 milhões de dispositivos móveis, pouco mais de 1 dispositivo móvel por habitante. Já na 31ª edição\cite{pesquisati31}, de junho de 2020, esse valor passou para de 342 milhões de dispositivos móveis, equivalente a aproximadamente 1,6 dispositivos por habitante. De acordo com dados do sensos realizados pelo IBGE, a população idosa(pessoas adultas acima de 65 anos) no Brasil cresceu de 8.9 milhões em 2017 para 9.8 milhões em 2020, um aumento de aproximadamente 11\% em 3 anos\cite{ibge}.

\par

Com o crescimento vertiginoso do uso de dispositivos móveis, há um aumento da demanda por aplicações que possam atender às mais diversas necessidades das pessoas que os utilizam. Levando em consideração os dados citados previamente, é plausível considerar que boa parte dos novos dispositivos pertençam à pessoas da terceira idade, que comumente apresentam dificuldade para executar diversas ações que são essenciais na interação de dispositivos móveis, em especial, os com tela tátil. Além dos fatores mais comuns, pessoas idosas apresentam problemas motores, de visão e cognitivos, que tornam ainda mais difíceis essas interações. Segundo Claudia Zapata \emph{et al}, “O desenvolvimento de aplicações para dispositivo móveis se tornou uma forma de melhorar a qualidade de vida para idosos, já que é possível aplicar a vários setores como a medicina, por exemplo”\cite{elderlyMainChallenges}.

\par

Este projeto foi idealizado a partir de um trabalho anterior: \textit{Um framework para facilitar as interações entre dispositivos móveis e pessoas idosas}\cite{tesedamaris}. Em sua tese, Dâmaris buscou identificar as ações em dispositivos móveis que os idosos enfrentam maior dificuldade para executar. Para conseguir identificar algumas dessas dificuldades, foram realizados testes em um trabalho conjunto com o \emph{Programa da Terceira Idade}, um projeto da Prefeitura Municipal de Viçosa - MG. Através destes testes foram identificadas algumas das ações que os idosos mais apresentam dificuldade em executar: o movimento de pinça, movimento de rotação e digitação.

\par

Com o resultado das pesquisas em mãos, foi então criado o \textit{ElderlyFrame}\cite{elderlyframe}, um \textit{framework} de interface de usuário nativo para \textit{Android}. O projeto contém um conjunto de componentes visuais com o objetivo de facilitar ações que geralmente exigem mais coordenação motora e acuidade visual. A seguir, temos os exemplos destes componentes que foram implementados.

\begin{figure}[H]
	\begin{center}
		\includegraphics[height=0.4\linewidth]{images/touchable-zoom.png}
	\end{center}
	\caption[Componente \textit{TouchZoom} do \textit{ElderlyFrame}]{Componente de Zoom por toque do \textit{ElderlyFrame}}
	\legend{Fonte: \citeauthor{tesedamaris} \cite{tesedamaris}}
	\label{fig:touchableZoom}
\end{figure}

\begin{figure}[H]
	\begin{center}
		\includegraphics[height=0.5\linewidth]{images/zoom-bar.png}
	\end{center}
	\caption[Componente \textit{SeekBarZoom} do \textit{ElderlyFrame}]{Componente de Zoom por barra do \textit{ElderlyFrame}}
	\legend{Fonte: \citeauthor{tesedamaris} \cite{tesedamaris}}
	\label{fig:zoomBar}
\end{figure}

\begin{figure}[H]
	\begin{center}
		\includegraphics[height=0.5\linewidth]{images/rotation.png}
	\end{center}
	\caption[Componente \textit{SimpleRotation} do \textit{ElderlyFrame}]{Componente de rotação com apenas um dedo do \textit{ElderlyFrame}}
	\legend{Fonte: \citeauthor{tesedamaris} \cite{tesedamaris}}
	\label{fig:rotation}
\end{figure}

\begin{figure}[H]
	\begin{center}
		\includegraphics[height=0.5\linewidth]{images/speech-to-text.png}
	\end{center}
	\caption[Componente \textit{SpeechToText} do \textit{ElderlyFrame}]{Componente de conversão de voz para texto (\textit{speech-to-text}) do \textit{ElderlyFrame}}
	\legend{Fonte: \citeauthor{tesedamaris} \cite{tesedamaris}}
	\label{fig:speechToText}
\end{figure}

\chapter{Justificativa}\label{sec:justificativa}

Atualmente existem implementações para \textit{React Native} de ferramentas que facilitam a presença de recursos de acessibilidade em aplicações para dispositivos móveis. Porém, estas ferramentas utilizam apenas recursos já existentes, funcionando apenas como facilitadoras. Este é um diferencial do \textit{ElderlyFrame}, e consequentemente deste projeto, que traz uma nova abordagem para interações que pessoas idosas apresentam dificuldade em executar.

\par

Em pesquisas realizadas para elaboração deste projeto, somente foram encontrados artigos e teses que abordavam a teoria da acessibilidade para idosos. Conteúdos que abordavam a falta de inclusão digital da terceira idade,  a falta de recursos de acessibilidade em aplicações para dispositivos móveis e afins. Porém, foi notada uma carência de projetos que de fato produzissem algo para mudar este cenário. O \textit{ElderlyFrame} foi construído com este propósito, e essa também é uma das motivações desta pesquisa.

\par

Como dito anteriormente, o \textit{ElderlyFrame} é um \textit{framework} de interface de usuário nativo para \textit{Android}, e este é um fator limitante da sua utilização. Ampliar o impacto do projeto e abranger uma maior parcela do cenário de desenvolvimento para dispositivos móveis são fatores motivadores para este projeto. Para isto, decidiu-se por focar em uma tecnologia de desenvolvimento multiplataforma. Tecnologias de desenvolvimento multiplataforma permitem o desenvolvimento de aplicações para diferentes ambientes, como \emph{Android} e \emph{iOS}, utilizando uma mesma base de código. A tecnologia escolhida para análise nesta pesquisa foi \textit{React Native}, pois é utilizado em aplicações com grande representatividade, como \textit{Instagram}, \textit{AirBnb} e \textit{Uber}.

\begin{figure}[H]
	\begin{center}
		\includegraphics[width=1\linewidth]{images/github-compare.png}
	\end{center}
	\caption[Comparativo de estatísticas dos repositórios de tecnologias de desenvolvimento multiplataforma]{Comparação entre as estatísticas dos repositórios de tecnologias de desenvolvimento multiplataforma: \textit{React Native}, \textit{Flutter}, \textit{Ionic} e \textit{Kivy}}
	\label{fig:githubCompare}
	\legend{Fonte: \textit{GitHub Compare}. Disponível em \url{https://www.githubcompare.com/facebook/react-native+flutter/flutter+ionic-team/ionic-framework+kivy/kivy/}}
\end{figure}

A imagem acima mostra que o repositório do \textit{React Native}, no \textit{GitHub}, tem a maior quantidade de \textit{forks} e o 2º maior número de \textit{stars} e \textit{issues}, quando comparado com outras tecnologias de desenvolvimento multiplataforma, a saber: \textit{Flutter},\textit{Ionic} e \textit{Kivy}.

Além das estatísticas apresentadas no parágrafo anterior, foi usada como uma base uma pesquisa realizada pela \textit{Statista}, que aborda o uso de \textit{frameworks} para desenvolvimento multiplataforma. Como pode ser observado, o \textit{React Native} representa aproximadamente 42\% do uso em aplicações. Em um cenário com tantas opções, 42\% é um valor significativamente grande e que poderia ampliar o impacto dos resultados deste projeto.

\begin{figure}[H]
	\begin{center}
		\includegraphics[height=.5\linewidth]{images/mobile-frameworks-statista.png}
	\end{center}
	\caption[Comparativo de uso de frameworks para desenvolvimento multiplataforma]{Comparação entre as estatísticas de uso de \textit{frameworks} para desenvolvimento multiplataforma}
	\label{fig:statistaResearch}
	\legend{Fonte: \citeauthor{statista} \cite{statista}}
\end{figure}

\chapter{Objetivo}\label{sec:objetivos}

\section{Geral}

A proposta deste trabalho é implementar o \emph{ElderlyFrame} utilizando \emph{React Native}, uma tecnologia para desenvolvimento multiplataforma para dispositivos móveis, visando ampliar o impacto do mesmo e criar novas possibilidades de uso da ferramenta.

\section{Específicos}

Além da implementação, serão verificados os pontos em que o \emph{React Native} pode não ser suficiente para suprir as necessidades de implementação, buscar por novos recursos para serem adicionados ao escopo e criar uma aplicação que possibilite a validação do artefato a ser criado.


\chapter{Referencial Teórico}\label{sec:referencialTeorico}

Em uma conversa com o orientador deste projeto, Lucas Vegi, conheci a tese de mestrado desenvolvida pela Dâmaris Arruda: Um \textit{framework} para facilitar as interações entre dispositivos móveis e pessoas idosas\cite{tesedamaris}.

\par

Em seu trabalho, Dâmaris buscou elencar as ações em dispositivos móveis que os idosos enfrentam maior dificuldade para executar. Após testes realizados em um trabalho conjunto com o \emph{Programa da Terceira Idade}, um projeto da Prefeitura Municipal de Viçosa - MG, foram identificadas tais ações: movimento de pinça, movimento de rotação e digitação.

\par

Dadas as informações obtidas pelos testes realizados, foi então desenvolvido um \textit{framework} de interface de usuário nativo para \textit{Android}. O projeto contém um conjunto de componentes visuais com o objetivo de facilitar ações que geralmente exigem mais coordenação motora e acuidade visual.

\chapter{Metodologia}\label{sec:metodologia}

Para dar início a este trabalho, primeiro foi necessário compreender o funcionamento do \textit{ElderlyFrame}. Para tal, foi criada uma aplicação para testes, em que foram utilizados todos os componentes implementados pelo framework. Dessa forma, foi possível identificar o funcionamento esperado de cada um, para serem posteriormente implementados no \textit{react-native-accessibility-elderly}, nome dado ao artefato produzido ao fim produzido através deste trabalho.

\section{O \textit{ElderlyFrame}}

Enquanto analisava os códigos da implementação do do \textit{ElderlyFrame}, pude identificar alguns problemas. Como o foco principal do trabalho da Dâmaris foi a realização da pesquisa, imagino que isso possa ter impactado o resultado final da implementação.

Como a primeira versão do \textit{ElderlyFrame} foi publicada em 2019, 2 anos antes do desenvolvimento deste presente trabalho, fez-se necessário realizar algumas mudanças, sendo elas:

\begin{itemize}
	\item Atualização da ferramenta \textit{gradle} para a versão 5.1.1
	\item Atualização da \textit{minSdkVersion} para a versão 16
	\item Atualização da \textit{compileSdkVersion} para a versão 30
	\item Mudança repositório central de \textit{JCenter} para \textit{MavenCentral}
	\item Mudança do pacote \textit{android.support} para o pacote \textit{androidx}
	\item Correção na implementação do ícone do componente \textit{SpeechText}
\end{itemize}

Para fins de registro das modificações realizadas, foi feito um \textit{fork} do projeto da \textit{ElderlyFrame} na plataforma \textit{GitHub}, que pode ser acessado no link abaixo: \linebreak

\href{https://github.com/reisdev/elderlyframe/}{https://github.com/reisdev/elderlyframe}.


\section{O framework \textit{react-native-accessibility-elderly}}

\par

A ideia inicial era implementar o framework de forma nativa. Ou seja, os trechos de código seriam escritos separadamente para cada plataforma. Para \textit{iOS}, usando \textit{Swift} ou \textit{Objective-C}, e para \textit{Android}, usando \textit{Java} ou \textit{Kotlin}. Porém, a tecnologia utilizada, \textit{React Native}, ainda apresenta algumas limitações com relação à implementação de módulos nativos, e isso levou à decisão de criar os componentes utilizando apenas \textit{Javascript} e \textit{Typescript}, linguagens usadas pela própria tecnologia.

\subsection{Recursos de terceiros}

A criação dos componentes do framework exigiu que algumas bibliotecas de terceiros fossem utilizadas, pois as mesmas forneciam recursos que demandariam muito esforço para serem implementados, que não caberia dentro dos prazos do projeto. Abaixo, temos a especificação de cada biblioteca e os recursos que as mesmas oferecem:

\begin{itemize}
	\item \textit{@react-native-voice/voice}: Reconhecimento de voz e conversão de voz para texto
	\item \textit{@react-native-community/slider}: Componente \textit{Slider} para \textit{React Native}
	\item \textit{react-native-vector-icons}: Conjunto de ícones variados
	\item \textit{react-native-gesture-handler}: Reconhecimento de gestos na tela de um dispositivo móvel.
\end{itemize}

\subsection{Os desafios}

Uma das maiores barreiras encontradas ao desenvolver este framework foi o próprio \textit{React Native}. Apesar de ser uma tecnologia madura, ainda apresenta algumas limitações com relação ao desenvolvimento de pacotes.

\par

A ferramenta \textit{yarn}, alternativa ao \textit{npm}, foi usada para criar um ambiente de desenvolvimento, pois ela permite que uma aplicação e uma biblioteca sejam criadas num mesmo ambiente, compartilhando a instalação de pacotes e assim, reduzindo o espaço em disco necessário. Nesse sentido, o uso do \textit{yarn} foi essencial na hora de acelerar o processo de desenvolvimento e também de testes do framework.

\subsection{Publicação de pacote}

O \textit{react-native-accessibility-elderly} será um publicado em forma de pacote usando a ferramenta \textit{npm}, que é um gerenciador de pacotes. Sendo assim, qualquer pessoa que deseje criar uma aplicação para dispositivos móveis com recursos de acessibilidade para idosos, poderá utilizar os componentes deste framework através da instalação do pacote usando o \textit{npm}.

A tecnologia \textit{React Native} exige que pacotes com implementações nativas passem por um processo de \textit{link}, que consiste em referenciar as respectivas implementações para que as aplicações que as usam possam encontrar e utilizar seus recursos.

\chapter{Resultados}\label{sec:resultados}

\chapter{Trabalhos Futuros}\label{sec:futuro}

\chapter{Cronograma}\label{sec:cronograma}

Para execução deste projeto, foram estabelecidas as atividades a serem realizadas e os meses que em que irão ocorrer durante o período deste projeto. O quadro a seguir apresenta essa relação:

\par

\begin{table}[htbp]
	\centering
	\caption[Cronograma mensal]{Cronograma do Projeto em Meses}
	\label{tab:cronogramaMensal}
	\begin{tabular}{lcccccccc} %|c|c|c|c|c|c|c|c|c|c|c|c
		\toprule
		\textbf{Atividade} & \textbf{1} & \textbf{2} & \textbf{3} & \textbf{4} & \textbf{5} & \textbf{6} & \textbf{7} & \textbf{8} \\
		\midrule
		Análise do Tema    & $\bullet$  & $\bullet$  &            &            &            &            &            &            \\
		Revisão Literária  &            & $\bullet$  & $\bullet$  &            &            & $\bullet$  & $\bullet$  &            \\
		Implementação      &            & $\bullet$  & $\bullet$  & $\bullet$  & $\bullet$  & $\bullet$  & $\bullet$  &            \\
		Documentação       &            &            & $\bullet$  & $\bullet$  & $\bullet$  & $\bullet$  & $\bullet$  & $\bullet$  \\
		Monografia         &            &            & $\bullet$  & $\bullet$  & $\bullet$  & $\bullet$  & $\bullet$  & $\bullet$  \\
		Testes             &            &            &            &            &            & $\bullet$  & $\bullet$  & $\bullet$  \\
		\bottomrule
	\end{tabular}
	\fonte{Próprio Autor}
\end{table}

As demandas de tempo(em horas) estimadas para orientação, desenvolvimento e revisão da literatura, do projeto escrito e demais artefatos produzidos são:

\par

\begin{table}[htpb]
	\centering
	\caption[Cronograma em horas]{Estimativa de tempo para cada atividade}
	\label{tab:cronogramaHoras}
	\begin{tabular}{lcc}
		\toprule
		\textbf{Item} & \textbf{Custeio} & \textbf{Tempo(horas)} \\
		\midrule
		Orientação    & Governo Federal  & 40                    \\
		Projeto       & Próprio          & 300                   \\
		Revisão       & Próprio          & 20                    \\
		\bottomrule
		Total         &                  & 360
	\end{tabular}
	\fonte{Próprio autor}
\end{table}

\chapter{Orçamento}\label{sec:orcamento}

Estão orçados para este projeto os itens contidos na tabela abaixo, acompanhados de seus respectivos financiadores e valores:

\par

\begin{table}[htpb]
	\centering
	\caption[Orçamento]{Orçamento financeiro}
	\label{tab:orcamento}
	\begin{tabular}{lccc}
		\toprule
		\textbf{Item}                         & \textbf{Custeio} & \textbf{Custo} \\
		\midrule
		Computador Pessoal                    & Próprio          & R\$ 5000       \\
		Conexão com a \textit{Internet}       & Próprio          & R\$ 900        \\
		Licença para publicação na Play Store & Próprio          & ~R\$140        \\
		\bottomrule
		Total                                 &                  & R\$6040        \\
	\end{tabular}%
	\fonte{Próprio Autor}
\end{table}%

% \chapter{Resultados Esperados}\label{sec:resultEsperados}

% ----------------------------------------------------------
% ELEMENTOS PÓS-TEXTUAIS
% ----------------------------------------------------------
\postextual

% Referências bibliográficas

\printbibliography[title={Referências Bibliográficas}]

% Caso sejam necessários apêndices ou anexos em seu documento
% Use os ambientes abaixo

%% Apêndices
%
%% Inicia os apêndices
%\begin{apendicesenv}
%
%% Imprime uma página indicando o início dos apêndices
%\partapendices
%
%\chapter{Primeiro Apêndice}
%
%\chapter{Segundo Apêndice}
%
%\end{apendicesenv}
%
%
%% ----------------------------------------------------------
%% Anexos
%% ----------------------------------------------------------
%\begin{anexosenv}
%
%% Imprime uma página indicando o início dos anexos
%\partanexos
%
%\chapter{Primeiro Anexo}
%\lipsum[30]
%
%\chapter{Segundo Anexo}
%\lipsum[31]
%
%\end{anexosenv}

\end{document}